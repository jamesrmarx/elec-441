\documentclass{article}
\usepackage[utf8]{inputenc}
\usepackage[margin=1in]{geometry}
\usepackage{wrapfig}
\usepackage{graphicx}
\usepackage[labelfont=bf]{caption}
\usepackage{subcaption}
\usepackage{setspace}
\usepackage{circuitikz}
\usepackage{amsmath}
\usepackage{amssymb}
\usepackage{float}
\usepackage{booktabs}
\usepackage{enumitem}
\usepackage{minted}

\title{ELEC 441 Assignment 2}
\author{James Marx \\ \\ 80562549}
\date{February 18, 2025}

\begin{document}

\maketitle
\newpage

\newpage

\section*{P1}

\begin{enumerate}[label=\alph*)]
    \item Continuous-time system:

    \[
     \dot x(t) = 
    \begin{bmatrix}
        -1 & 0 & 0 \\
        0 & -1 & 1 \\
        0 & -2 & -1
    \end{bmatrix}
    x(t)
    \]

    \begin{enumerate}[label=\roman*)]
        \item Eigenvalue Criteria:

        We begin by computing the eigenvalues of $A$

        \[ \det(A - \lambda I) = 0\]

        \[
            \det \begin{bmatrix}
                -1 - \lambda & 0 & 0 \\
                0 & -1 - \lambda & 1 \\
                0 & -2 & -1 - \lambda
            \end{bmatrix} 
            = 0
        \]

        \begin{align*}
            &= (-1- \lambda)\left((-1 - \lambda)^2 - (-2)\right) - 0 + 0 = 0 \\
            &= (-1 - \lambda)(\lambda^2 + 2\lambda + 3) = 0 \\
        \end{align*}

        \[\lambda_1 = -1, \quad \lambda_2 = -1 + j\sqrt{2}, \quad \lambda_3 = -1 - j\sqrt{2}\]

        $\boxed{\text{This system is asympototically stable as $\text{Re}\{\lambda_i\} < 0, \forall i$}}$

        \item Lyapunov Theorem:

        For the CT system, we solve for $P$ in the following equation

        \[A^T P + PA = -Q\]

        Taking $Q$ to be the identity marix

        \[
        \begin{bmatrix}
            -1 & 0 & 0 \\
            0 & -1 & -2 \\
            0 & 1 & -1
        \end{bmatrix} 
        \begin{bmatrix}
            p_1 & p_2 & p_3 \\
            p_2 & p_4 & p_5 \\
            p_3 & p_5 & p_6 \\
        \end{bmatrix}
        +
        \begin{bmatrix}
            p_1 & p_2 & p_3 \\
            p_2 & p_4 & p_5 \\
            p_3 & p_5 & p_6 \\
        \end{bmatrix}
        \begin{bmatrix}
            -1 & 0 & 0 \\
            0 & -1 & 1 \\
            0 & -2 & -1
        \end{bmatrix} 
        =
        \begin{bmatrix}
            -1 & 0 & 0 \\
            0 & -1 & 0 \\
            0 & 0 & -1
        \end{bmatrix} 
        \]

        \[
        \begin{bmatrix}
            -2p_1 & -2p_2 - 2p_3 & p_2 - 2p_3 \\
            -2p_2 & -2p_4 - 4p_5 & p_4 - 2p_5 - 2p_6 \\
            p_2 - 2p_3 & p_4 - 2p_5 & 2p_5 - 2p_6
        \end{bmatrix} 
        =
        \begin{bmatrix}
            -1 & 0 & 0 \\
            0 & -1 & 0 \\
            0 & 0 & -1
        \end{bmatrix} 
        \]
        \begin{align*}
            p_1 &= \boxed{\frac{1}{2}}
        \end{align*}
        \begin{align*}
            -2p_4 - 4p_5 &= -1 \\
            -2p_4 &= -1 + 4p_5 \\
            p_4 &= \frac{1}{2} - 2p_5
        \end{align*}

        \begin{align*}
            2p_5 - 2p_6 &= -1 \\
            -2p_6 &= -1 - 2p_5 \\
            p_6 &= \frac{1}{2} + p_5
        \end{align*}

        \begin{align*}
            p_4 - 2p_5 - 2p_6 &= 0 \\
            \frac{1}{2} - 2p_5 - 2p_5 - 2\left(\frac{1}{2} + p_5\right) &= 0 \\
            \frac{1}{2} - 4p_5 - 1 - 2p_5 &= 0 \\
            \frac{1}{2} -6p_5 - 1 &= 0 \\
            -\frac{1}{2} - 6p_5 &= 0 \\
        \end{align*}

        \begin{align*}
            p_5 &= \boxed{-\frac{1}{12}}
        \end{align*}

        \begin{align*}
            p_4 &= \boxed{\frac{2}{3}}
        \end{align*}

        \begin{align*}
            p_6 &= \boxed{\frac{5}{12}}
        \end{align*}

        \begin{align*}
            p_2 = p_3 &= \boxed{0}
        \end{align*}

        \[
        P = 
        \begin{bmatrix}
            \frac{1}{2} & 0 & 0 \\
            0 & \frac{2}{3} & -\frac{1}{12} \\
            0 & -\frac{1}{2} & \frac{5}{12}
        \end{bmatrix}
        \]

        All diagonal entries are $>$ 0 so $P$ may be positive definite

        Check principle minors
        \begin{align*}
            A_1 &= \frac{1}{2} > 0 \\
            A_2 &= \det
            \begin{bmatrix}
            \frac{1}{2} & 0 \\
            0 & \frac{2}{3}   
            \end{bmatrix}
            = \frac{1}{2} \times \frac{2}{3} > 0 \\
            A_3 &= \det
            \begin{bmatrix}
                \frac{1}{2} & 0 & 0 \\
                0 & \frac{2}{3} & -\frac{1}{12} \\
                0 & -\frac{1}{12} & \frac{5}{12} 
            \end{bmatrix}
            = \frac{1}{2}\det
            \begin{bmatrix}
                \frac{2}{3} & -\frac{1}{12} \\
                -\frac{1}{12} & \frac{5}{12}
            \end{bmatrix}
            + 0 + 0 \\
            &= \frac{1}{2}\left(\frac{2}{3} \times  \frac{5}{12} - \frac{1}{12} \times \frac{1}{12}\right) > 0
        \end{align*}

        All principle minors are greater than 0 so $P$ is positive definite.

        $\boxed{\text{The system is asympototically stable since $P > 0$}}$
    \end{enumerate}

    \item Discrete-time system:

    \[
    x[k + 1] =
    \begin{bmatrix}
    0 & -1 \\
    \frac{1}{2} & 0 
    \end{bmatrix} 
    x[k]
    \]

    \begin{enumerate}[label=\roman*)]
        \item Eivenvalue Criteria:

        We begin by computing the eigenvalues of $A$

        \[ \det(A - \lambda I) = 0\]

        \[
        \det
        \begin{bmatrix}
            -\lambda & -1 \\
            \frac{1}{2} & -\lambda
        \end{bmatrix}
        = \lambda^2 + \frac{1}{2} = 0
        \]

        \[\lambda_1 = j\frac{\sqrt{2}}{2}, \quad \lambda_2 = \frac{\sqrt{2}}{j2}\]

        \[ |\lambda_1| = |\lambda_2| = \frac{\sqrt{2}}{2} < 1\]

        $\boxed{\text{This system is asympototically stable as $|\lambda_i| < 1, \forall i$}}$

    \item Lyapunov Theorem:
    
    For the DT system, we solve for $P$ in the following equation

    \[A^T P A - P = -Q\]

    Taking Q to be the identity matrix

    \[
    \begin{bmatrix}
        0 & \frac{1}{2} \\
        -1 & 0
    \end{bmatrix}
    \begin{bmatrix}
        p_1 & p_2 \\
        p_2 & p_3
    \end{bmatrix}
    \begin{bmatrix}
        0 & -1 \\
        \frac{1}{2} & 0
    \end{bmatrix}
    -
    \begin{bmatrix}
        p_1 & p_2 \\
        p_2 & p_3
    \end{bmatrix}
    = 
    \begin{bmatrix}
        -1 & 0 \\
        0 & -1
    \end{bmatrix}
    \]

    \[
    \begin{bmatrix}
        \frac{1}{4}p_3 - p_1 & -\frac{1}{2}p_3 - p_2 \\
        -\frac{1}{2}p_2 - p+2 & p_1 - p_3
    \end{bmatrix}
    =
    \begin{bmatrix}
        -1 & 0 \\
        0 & -1
    \end{bmatrix}
    \]

    \begin{align*}
        p_1 - p_3 &= -1 \\
        p_1 &= p_3 - 1
    \end{align*}

    \begin{align*}
        \frac{1}{4}p_3 - p_1 &= -1 \\
        \frac{1}{4}p_3 - p_3 + 1 &= -1 \\
        -\frac{3}{4}p_3 &= -2
    \end{align*}

    \begin{align*}
        p_3 &= \boxed{\frac{8}{3}}
    \end{align*}

    \begin{align*}
        p_1 &= \boxed{\frac{5}{3}}
    \end{align*}

    \begin{align*}
        p_2 &= \boxed{0}
    \end{align*}

    \[
    P = 
    \begin{bmatrix}
        \frac{5}{3} & 0 \\
        0 & \frac{8}{3}
    \end{bmatrix} 
    \]

    $P$ is diagonal with positive entries and is therefore positive definite.

    $\boxed{\text{The system is asympototically stable since $P > 0$}}$
    \end{enumerate}
\end{enumerate}

\section*{P2}

\begin{enumerate}[label=\alph*)]
    \item $M_1 = M_2 = k_1 = k_2 = 1$

    \[
    A = 
    \begin{bmatrix}
        0 & 1 & 0 & 0 \\
        -2 & 0 & 1 & 0 \\
        0 & 0 & 0 & 1 \\
        1 & 0 & -1 & 0
    \end{bmatrix}
    , \quad B = 
    \begin{bmatrix}
        0 \\
        1 \\
        0 \\
        0
    \end{bmatrix}
    \]
     
    \[ 
    C = 
    \begin{bmatrix}
        -1 & 0 & 1 & 0 \\
        -2 & 0 & 1 & 0 \\
        1 & 0 & -1 & 0
    \end{bmatrix} 
    , \quad D = 
    \begin{bmatrix}
        0 \\
        1 \\
        0
    \end{bmatrix}
    \]

    \begin{enumerate}[label=\roman*)]
        \item Controllability
        
        \[
        \mathcal{C} = 
        \begin{bmatrix}
            B & AB & \cdots & A^{n-1}B
        \end{bmatrix}
        \in \mathbb{R}^{n \times nm}
        \]

        \[
        A^2 = 
        \begin{bmatrix}
            0 & 1 & 0 & 0 \\
            -2 & 0 & 1 & 0 \\
            0 & 0 & 0 & 1 \\
            1 & 0 & -1 & 0
        \end{bmatrix}
        \begin{bmatrix}
            0 & 1 & 0 & 0 \\
            -2 & 0 & 1 & 0 \\
            0 & 0 & 0 & 1 \\
            1 & 0 & -1 & 0
        \end{bmatrix}
        =
        \begin{bmatrix}
            -2 & 0 & 1 & 0 \\
            0 & -2 & 0 & 1 \\
            1 & 0 & -1 & 0 \\
            0 & 1 & 0 & -1
        \end{bmatrix}
        \]

        \[
        A^3 = 
        \begin{bmatrix}
            -2 & 0 & 1 & 0 \\
            0 & -2 & 0 & 1 \\
            1 & 0 & -1 & 0 \\
            0 & 1 & 0 & -1
        \end{bmatrix}
        \begin{bmatrix}
            0 & 1 & 0 & 0 \\
            -2 & 0 & 1 & 0 \\
            0 & 0 & 0 & 1 \\
            1 & 0 & -1 & 0
        \end{bmatrix}
        = 
        \begin{bmatrix}
            0 & -2 & 0 & 1 \\
            5 & 0 & -3 & 0 \\
            0 & 1 & 0 & -1 \\
            3 & 0 & 2 & 0
        \end{bmatrix}
        \]

    We can determine from inspection, that pre-multiplying $B$ with $A^n$ will result in the second column of $A^n$

    \[
    \mathcal{C} = 
    \begin{bmatrix}
        0 & 1 & 0 & -2 \\
        1 & 0 & -2 & 0 \\
        0 & 0 & 0 & 1 \\
        0 & 0 & 1 & 0
    \end{bmatrix}
    \]

    We can easily determine the rank of $\mathcal{C}$ by re-arranging the rows

    \[
    \begin{bmatrix}
        0 & 1 & 0 & -2 \\
        1 & 0 & -2 & 0 \\
        0 & 0 & 0 & 1 \\
        0 & 0 & 1 & 0
    \end{bmatrix}
    \sim
    \begin{bmatrix}
        1 & 0 & -2 & 0 \\
        0 & 1 & 0 & -2 \\
        0 & 0 & 1 & 0 \\
        0 & 0 & 0 & 1
    \end{bmatrix}
    \Rightarrow \operatorname{rank}(\mathcal{C}) = 4
    \]

    $\boxed{\text{Since the Controllability matrix has full rank, this system is controllable}}$

    \item Observability
    
    \[
    \mathcal{O} = 
    \begin{bmatrix}
        C \\
        CA \\
        \vdots \\
        CA^{n-1}
    \end{bmatrix}
    \]

    \[
    CA = 
    \begin{bmatrix}
        -1 & 0 & 1 & 0 \\
        -2 & 0 & 1 & 0 \\
        1 & 0 & -1 & 0
    \end{bmatrix} 
    \begin{bmatrix}
        0 & 1 & 0 & 0 \\
        -2 & 0 & 1 & 0 \\
        0 & 0 & 0 & 1 \\
        1 & 0 & -1 & 0
    \end{bmatrix}
    =
    \begin{bmatrix}
        0 & -1 & 0 & 1 \\
        0 & -2 & 0 & 1 \\
        0 & 1 & 0 & -1
    \end{bmatrix}
    \]

    \[
    CA^2 = 
    \begin{bmatrix}
        -1 & 0 & 1 & 0 \\
        -2 & 0 & 1 & 0 \\
        1 & 0 & -1 & 0
    \end{bmatrix} 
    \begin{bmatrix}
        -2 & 0 & 1 & 0 \\
        0 & -2 & 0 & 1 \\
        1 & 0 & -1 & 0 \\
        0 & 1 & 0 & -1
    \end{bmatrix}
    =
    \begin{bmatrix}
        3 & 0 & -2 & 0 \\
        5 & 0 & -3 & 0 \\
        -3 & 0 & 2 & 0
    \end{bmatrix}
    \]

    \[
    CA^3 = 
    \begin{bmatrix}
        -1 & 0 & 1 & 0 \\
        -2 & 0 & 1 & 0 \\
        1 & 0 & -1 & 0
    \end{bmatrix} 
    \begin{bmatrix}
        0 & -2 & 0 & 1 \\
        5 & 0 & -3 & 0 \\
        0 & 1 & 0 & -1 \\
        3 & 0 & 2 & 0
    \end{bmatrix}
    =
    \begin{bmatrix}
        0 & 3 & 0 & -2 \\
        0 & 5 & 0 & -3 \\
        0 & -3 & 0 & 2
    \end{bmatrix}
    \]

    \[
    \mathcal{O} =
    \begin{bmatrix}
        -1 & 0 & 1 & 0 \\
        -2 & 0 & 1 & 0 \\
        1 & 0 & -1 & 0 \\
        0 & -1 & 0 & 1 \\
        0 & -2 & 0 & 1 \\
        0 & 1 & 0 & -1 \\
        3 & 0 & -2 & 0 \\
        5 & 0 & -3 & 0 \\
        -3 & 0 & 2 & 0 \\
        0 & 3 & 0 & -2 \\
        0 & 5 & 0 & -3 \\
        0 & -3 & 0 & 2
    \end{bmatrix}
    \]

    We can conlcude the rank of $\mathcal{O}$ is 4 if we can find a subset of the rows with rank 4

    \[
    \begin{bmatrix}
        5 & 0 & -3 & 0 \\
        -3 & 0 & 2 & 0 \\
        0 & 3 & 0 & -2 \\
        0 & 5 & 0 & -3 \\
    \end{bmatrix}
    \sim
    \begin{bmatrix}
        1 & 0 & -\frac{3}{5} & 0 \\
        1 & 0 & -\frac{2}{3} & 0 \\
        0 & 1 & 0 & -\frac{2}{3} \\
        0 & 1 & 0 & -\frac{3}{5}
    \end{bmatrix}
    \sim
    \begin{bmatrix}
        1 & 0 & -\frac{3}{5} & 0 \\
        0 & 0 & 1 & 0 \\
        0 & 1 & 0 & -\frac{2}{3} \\
        0 & 1 & 0 & -\frac{3}{5}
    \end{bmatrix}
    \]
    \[
    \sim
    \begin{bmatrix}
        1 & 0 & -\frac{3}{5} & 0 \\
        0 & 1 & 0 & -\frac{2}{3} \\
        0 & 0 & 1 & 0 \\
        0 & 1 & 0 & -\frac{3}{5}
    \end{bmatrix}
    \sim
    \begin{bmatrix}
        1 & 0 & -\frac{-3}{5} & 0 \\
        0 & 1 & 0 & -\frac{2}{3} \\
        0 & 0 & 1 & 0 \\
        0 & 0 & 0 & 1
    \end{bmatrix}
    = \operatorname{rank}(\mathcal{O}) = 4
    \]

    $\boxed{\text{Since $\mathcal{O}$ has a rank of 4, the system is observable}}$


    \end{enumerate}

    \item $R = C = 1$

    \[
    A = 
    \begin{bmatrix}
        -3 & 1 & 0 \\
        1 & -2 & 1 \\
        0 & 1 & -3
    \end{bmatrix}
    ,\quad B = 
    \begin{bmatrix}
        2 & 0 \\
        0 & 0 \\
        0 & 2
    \end{bmatrix}
    , \quad C = 
    \begin{bmatrix}
        0 & 0 & 1
    \end{bmatrix}
    \]


    \begin{enumerate}[label=\roman*)]
        \item Controllability
 
        \[
        \mathcal{C} = 
        \begin{bmatrix}
            B & AB & \cdots & A^{n-1}B
        \end{bmatrix}
        \in \mathbb{R}^{n \times nm}
        \]

        \[
        A^2 = 
        \begin{bmatrix}
        -3 & 1 & 0 \\
        1 & -2 & 1 \\
        0 & 1 & -3
        \end{bmatrix} 
        \begin{bmatrix}
        -3 & 1 & 0 \\
        1 & -2 & 1 \\
        0 & 1 & -3
        \end{bmatrix} 
        =
        \begin{bmatrix}
            10 & -5 & 1 \\
            -5 & 6 & -4 \\
            1 & -4 & 5
        \end{bmatrix}
        \]

        Pre-multiplying $B$ with $A^n$ will result in a matrix with twice the first column of $A^n$ and twice the third column of $A^n$

        \[
        AB =
        \begin{bmatrix}
            -6 & 0 \\
            2 & 2 \\
            0 & -6
        \end{bmatrix}
        \]

        \[
        A^2B =
        \begin{bmatrix}
            20 & 2 \\
            -10 & -8 \\
            2 & 10
        \end{bmatrix}
        \]

        \[
        \mathcal{C} = 
        \begin{bmatrix}
            2 & 0 & -6 & 0 & 20 & 2 \\
            0 & 0 & 2 & 2 & -10 & -8 \\
            0 & 2 & 0 & -6 & 2 & 10 
        \end{bmatrix} 
        \]

        $\mathcal{C}$ will have rank 3 if we can find 3 linearly independent columns

        \[
        \begin{bmatrix}
            2 & 0 & -6 \\
            0 & 0 & 2 \\
            0 & 2 & 0
        \end{bmatrix}
        \sim
        \begin{bmatrix}
            1 & 0 & -3 \\
            0 & 1 & 0 \\
            0 & 0 & 1
        \end{bmatrix}
        \Rightarrow \operatorname{rank}(\mathcal{C}) = 3
        \]

    $\boxed{\text{Since the Controllability matrix has rank 3, this system is controllable}}$

    \item Observability
     
    \[
    \mathcal{O} = 
    \begin{bmatrix}
        C \\
        CA \\
        \vdots \\
        CA^{n-1}
    \end{bmatrix}
    \]

    Pre-multiplying $A^n$ with $C$ will result in the third row of $A^n$.

    \[
    \mathcal{O} = 
    \begin{bmatrix}
        0 & 0 & 1 \\
        0 & 1 & -3 \\
        1 & -4 & 5
    \end{bmatrix}
    = \operatorname{rank}(\mathcal{O}) = 3
    \]

    $\boxed{\text{Since $\mathcal{O}$ has a rank of 3, the system is observable}}$

    \end{enumerate}

    \item

    \[
    A = 
    \begin{bmatrix}
        0 & 1 & 0 & 0 \\
        \frac{g(M+m)}{lm} & 0 & 0 & 0 \\
        0 & 0 & 0 & 1 \\
        -\frac{mg}{M} & 0 & 0 & 0 
    \end{bmatrix}
    , \quad B = 
    \begin{bmatrix}
        0 \\
        -\frac{1}{lM} \\
        0 \\
        \frac{1}{M}
    \end{bmatrix}
    , \quad C = 
    \begin{bmatrix}
        0 & 0 & 1 & 0
    \end{bmatrix}
    \]

    \begin{enumerate}[label=\roman*)]
        \item Controllability

    \[
    A^2 = 
    \begin{bmatrix}
        0 & 1 & 0 & 0 \\
        \frac{g(M+m)}{lm} & 0 & 0 & 0 \\
        0 & 0 & 0 & 1 \\
        -\frac{mg}{M} & 0 & 0 & 0 
    \end{bmatrix}
    \begin{bmatrix}
        0 & 1 & 0 & 0 \\
        \frac{g(M+m)}{lm} & 0 & 0 & 0 \\
        0 & 0 & 0 & 1 \\
        -\frac{mg}{M} & 0 & 0 & 0 
    \end{bmatrix}
    = 
    \begin{bmatrix}
        \frac{g(M+m)}{lM} & 0 & 0 & 0 \\
        0 & \frac{g(M+m)}{lM} & 0 & 0 \\
        -\frac{mg}{M} & 0 & 0 & 0 \\
        0 & -\frac{mg}{M} & 0 & 0
    \end{bmatrix}
    \]

    \[
    A^3 = 
    \begin{bmatrix}
        0 & 1 & 0 & 0 \\
        \frac{g(M+m)}{lm} & 0 & 0 & 0 \\
        0 & 0 & 0 & 1 \\
        -\frac{mg}{M} & 0 & 0 & 0 
    \end{bmatrix}
    \begin{bmatrix}
        \frac{g(M+m)}{lM} & 0 & 0 & 0 \\
        0 & \frac{g(M+m)}{lM} & 0 & 0 \\
        -\frac{mg}{M} & 0 & 0 & 0 \\
        0 & -\frac{mg}{M} & 0 & 0
    \end{bmatrix}
    = 
    \begin{bmatrix}
        0 & \frac{g(M+m)}{lM} & 0 & 0 \\
        \left(\frac{g(M+m)}{lM}\right)^2 & 0 & 0 & 0 \\
        0 & -\frac{mg}{M} & 0 & 0 \\
        -\frac{mg^2(M+m)}{lM^2} & 0 & 0 & 0
    \end{bmatrix}
    \]

    \[
    AB = 
    \begin{bmatrix}
        -\frac{1}{lM} \\
        0 \\
        \frac{1}{M} \\
        0 
    \end{bmatrix}
    \]

    \[
    A^2B = 
    \begin{bmatrix}
        0 \\
        -\frac{g(M+m)}{(lM)^2} \\
        0 \\
        \frac{mg}{lM^2}
    \end{bmatrix}
    \]

    \[
    A^3B = 
    \begin{bmatrix}
        -\frac{g(M+m)}{(lM)^2} \\
        0 \\
        \frac{mg}{lM^2} \\
        0
    \end{bmatrix}
    \]

    \[
    \mathcal{C} = 
    \begin{bmatrix}
         0 & -\frac{1}{lM} & 0 & \frac{g(M+m)}{(lM)^2} \\
        -\frac{1}{lM} & 0 & -\frac{g(M+m)}{(lM)^2} & 0 \\
        0 & 0 & 0 & \frac{mg}{lM^2} \\
        \frac{1}{M} & 0 & \frac{mg}{lM^2} & 0 
    \end{bmatrix}
    \Rightarrow \operatorname{rank}(\mathcal{C}) = 4
    \]

    $\boxed{\text{Since $\mathcal{C}$ has a rank of 4, the system is controllable}}$


    \item Observability
    
    Pre-multiplying $A^n$ with $C$ will result in the fhird row of $A^n$
    \[
    \mathcal{O} = 
    \begin{bmatrix}
        0 & 0 & 1 & 0 \\
        0 & 0 & 0 & 1 \\
        -\frac{mg}{M} & 0 & 0 & 0 \\
        0 & -\frac{mg}{M} & 0 & 0
    \end{bmatrix}
    \Rightarrow \operatorname{rank}(\mathcal{O}) = 4
    \]

    $\boxed{\text{Since $\mathcal{O}$ has a rank of 4, the system is observable}}$
    
    \end{enumerate}
\end{enumerate}

\section*{P3}

\begin{enumerate}[label=\alph*)]
    \item 

\[
\left\{
    \begin{array}{l}
        \dot x(t) = 
        \begin{bmatrix}
            2 & -1 \\
            1 & 0 
        \end{bmatrix}
        x(t) + 
        \begin{bmatrix}
            1 \\
            0
        \end{bmatrix}
        u(t), \\
        y(t) = 
        \begin{bmatrix}
            1 & -1
        \end{bmatrix}
        x(t)
    \end{array}
\right.
\]

We first find the controllability matrix, which will simply be $B$ concatonated with the first column of $A$

\[
    \mathcal{C} =
    \begin{bmatrix}
        1 & 2 \\
        0 & 1
    \end{bmatrix}
    \Rightarrow \operatorname{rank}(\mathcal{C}) = 2
\]

This system is fully controllable.

The observability matrix will be $C$ and the second row of $A$ subtracted from the first

\[
    \mathcal{O} = 
    \begin{bmatrix}
        1 & -1 \\
        1 & -1    
    \end{bmatrix}
    \Rightarrow \operatorname{rank}(\mathcal{O}) = 1
\]

This system is not fully observable.

Since $\mathcal{C}$ has full rank, the Image space can be expressed as the span of the 2-D basis vectors

\[
\operatorname{Im}(\mathcal{C}) = \operatorname{span}
    \left\{
    \begin{bmatrix}
        1 \\
        0
    \end{bmatrix},
    \begin{bmatrix}
        0 \\
        1
    \end{bmatrix}
    \right\}
\]

We will next find the Kernel space of $\mathcal{O}$ which can easily be seen to be the line where $x_1 = x_2$

\[
    \operatorname{Ker}(\mathcal{O}) = \operatorname{span}
    \left\{
        \begin{bmatrix}
            1 \\
            1
        \end{bmatrix}
    \right\}
\]

We can now begin finding the bases of the subspaces needed to construct $T^{-1}$

\[
T^{-1} = 
\begin{bmatrix}
    T_{c o} & T_{c \bar o} & T_{\bar c o} & T_{\bar c \bar o}
\end{bmatrix}
\]

$T_{c \bar o}$ will be a basis for the intersection of $\operatorname{Im}(\mathcal{C})$ and $\operatorname{Ker}(\mathcal{O})$

\[
T_{c \bar o} = 
\begin{bmatrix}
    1 \\
    1
\end{bmatrix}
\]

$T_{\bar c o}$ will be the intersection of the orthogonal compliment of $\operatorname{Im}(\mathcal{C})$ and the orthgonal compliment of $\operatorname{Ker}(\mathcal{O})$, which is the zero subspace

\[
T_{\bar c o} = \mathbf{0}
\]

$T_{\bar c \bar o}$ will be the intersection of the orthogonal compliment of $\operatorname{Im}(\mathcal{C})$ and $\operatorname{Ker}(\mathcal{O})$, which again is the zero subspace

\[
T_{\bar c \bar o} = \mathbf{0}
\]

$T_{c o}$ will be the intersection of $\operatorname{Im}(\mathcal{C})$ and the orthgonal compliment of $\operatorname{Ker}(\mathcal{O})$

\[
T_{c o} = 
\begin{bmatrix}
    1 \\
    -1
\end{bmatrix}
\]

\[
T^{-1} = 
\begin{bmatrix}
    1 & 1 \\
    -1 & 1
\end{bmatrix}
\Rightarrow
T = 
\frac{1}{2}
\begin{bmatrix}
    1 & -1 \\
    1 & 1
\end{bmatrix}
\]

We can now perform the coordinate transformation

\begin{align*}
    \bar A &= T A T^{-1} \\
    &=
    \frac{1}{2}
    \begin{bmatrix}
        1 & -1 \\
        1 & 1
    \end{bmatrix}
    \begin{bmatrix}
        2 & -1 \\
        1 & 0
    \end{bmatrix}
    \begin{bmatrix}
        1 & 1 \\
        -1 & 1
    \end{bmatrix} \\
    &= \frac{1}{2}
    \begin{bmatrix}
        1 & -1 \\
        3 & -1 
    \end{bmatrix}
    \begin{bmatrix}
        1 & 1 \\
        -1 & 1
    \end{bmatrix} \\
    &= 
    \begin{bmatrix}
        1 & 0 \\
        2 & 1
    \end{bmatrix}
\end{align*}

\begin{align*}
    \bar B &= T B \\
    &= \frac{1}{2}
    \begin{bmatrix}
        1 \\
        1
    \end{bmatrix}
\end{align*}

\begin{align*}
    \bar C &= C T^{-1}  \\
    &=
    \begin{bmatrix}
        2 & 0
    \end{bmatrix}
\end{align*}

The new system after the coordinate transformation is

\[
\left\{
    \begin{array}{l}
        \dot z(t) = 
        \begin{bmatrix}
            1 & 0 \\
            2 & 1 
        \end{bmatrix}
        z(t) +  \frac{1}{2}
        \begin{bmatrix}
            1 \\
            1
        \end{bmatrix}
        u(t), \\
        y(t) = 
        \begin{bmatrix}
            2 & 0
        \end{bmatrix}
        z(t)
    \end{array}
\right.
\]

The new state vector $z$ is

\begin{align*}
    z & = T x \\
    &= \frac{1}{2}
    \begin{bmatrix}
        1 & -1 \\
        1 & 1
    \end{bmatrix}
    \begin{bmatrix}
        x_1 \\
        x_2
    \end{bmatrix} \\
    &= \frac{1}{2}
    \begin{bmatrix}
        x_1 - x_2 \\
        x_1 + x_2
    \end{bmatrix}
\end{align*}

The state which is controllable and observable is $\frac{1}{2}(x_1 - x_2)$

The state which is controllable and not observable is $\frac{1}{2}(x_1 + x_2)$

No states are not controllable and observable/not observable.

\item 

\[
\left\{
    \begin{array}{l}
        \dot x(t) = 
        \begin{bmatrix}
            1 & 0 & 0 \\
            0 & 1 & 1 \\
            1 & 0 & -1 
        \end{bmatrix}
        x(t) + 
        \begin{bmatrix}
            0 \\
            0 \\
            1
        \end{bmatrix}
        u(t), \\
        y(t) = 
        \begin{bmatrix}
            0 & 0 & 1
        \end{bmatrix}
        x(t)
    \end{array}
\right.
\]

We first find the controllability matrix, which will be $B$ concatonated with the thrid columns of $A^n$

\[
A^2 = 
\begin{bmatrix}
    1 & 0 & 0 \\
    0 & 1 & 1 \\
    1 & 0 & -1 
\end{bmatrix}
\begin{bmatrix}
    1 & 0 & 0 \\
    0 & 1 & 1 \\
    1 & 0 & -1 
\end{bmatrix}
=
\begin{bmatrix}
    1 & 1 & 0 \\
    1 & 1 & 0 \\
    0 & 0 & 1 
\end{bmatrix}
\]

\[
\mathcal{C} = 
\begin{bmatrix}
    0 & 0 & 0 \\
    0 & 1 & 0 \\
    1 & -1 & 1
\end{bmatrix}
\Rightarrow \operatorname{rank}(\mathcal{C}) = 2
\]

This system is partially controllable

The observability will be $C$ and the third rows of $A^n$

\[
\mathcal{O} = 
\begin{bmatrix}
    0 & 0 & 1 \\
    1 & 0 & -1 \\
    0 & 0 & 1
\end{bmatrix}
\Rightarrow \operatorname{rank}(\mathcal{O})= 2
\]

This system is partially observable

To find the Image space of $\mathcal{C}$ we row reduce and pick two linearly independent columns

\[
\begin{bmatrix}
    1 & -1 & 1 \\
    0 & 1 & 0 \\
    0 & 0 & 0
\end{bmatrix}
\sim
\begin{bmatrix}
    1 & 0 & 1 \\
    0 & 1 & 0 \\
    0 & 0 & 0
\end{bmatrix}
\]

\[
\operatorname{Im}(\mathcal{C}) = \operatorname{span}
\left\{
\begin{bmatrix}
    1 \\
    0 \\
    0
\end{bmatrix}
,
\begin{bmatrix}
    0 \\
    1 \\
    0
\end{bmatrix}
\right\}
\]

Now for $\operatorname{Ker}(\mathcal{O})$

\[
\begin{bmatrix}
    0 & 0 & 1 \\
    1 & 0 & -1 \\
    0 & 0 & 1
\end{bmatrix}
\begin{bmatrix}
    x_1 \\
    x_2 \\
    x_3
\end{bmatrix}
= \mathbf{0}
\]

\begin{align*}
    x_3 &= 0 \\
    x_1 - x_3 &= 0 \\
    x_2 & = \text{free}
\end{align*}

\[
\operatorname{Ker}(\mathcal{O}) = \operatorname{span}\left\{ 
    \begin{bmatrix}
        0 \\
        1 \\
        0
    \end{bmatrix}
\right\}
\]

We can now begin finding the bases of the subspaces needed to construct $T^{-1}$

\[
T^{-1} = 
\begin{bmatrix}
    T_{c o} & T_{c \bar o} & T_{\bar c o} & T_{\bar c \bar o}
\end{bmatrix}
\]



$T_{co}$ will be a basis for the intersection of $\operatorname{Im}(\mathcal{C})$ and the orthogonal compliment of $\operatorname{Ker}(\mathcal{O})$

\[
T_{co} = 
\begin{bmatrix}
    1 \\
    0 \\
    0
\end{bmatrix}
\]

$T_{c \bar o}$ will be the intersection of $\operatorname{Im}(\mathcal{C})$ and $\operatorname{Ker}(\mathcal{O})$

\[
T_{c \bar o} =
\begin{bmatrix}
    0 \\
    1 \\
    0
\end{bmatrix}
\]

$T_{\bar c o}$ will be the intersection of the orthogonal compliment of $\operatorname{Im}(\mathcal{C})$ and the orthgonal compliment of $\operatorname{Ker}(\mathcal{O})$

\[
T_{\bar c o} =
\begin{bmatrix}
    0 \\
    0 \\
    1
\end{bmatrix}
\]

$T_{\bar c \bar o}$ will be the intersection of the orthgonal compliment of  $\operatorname{Im}(\mathcal{C})$ and $\operatorname{Ker}(\mathcal{O})$

\[
T_{\bar c \bar o} = \mathbf{0}
\]

\[
T^{-1} = 
\begin{bmatrix}
    1 & 0 & 0 \\
    0 & 1 & 0 \\
    0 & 0 & 1
\end{bmatrix}
\Rightarrow
T = 
\begin{bmatrix}
    1 & 0 & 0 \\
    0 & 1 & 0 \\
    0 & 0 & 1
\end{bmatrix}
\]

After the transform, all  matricies are the same as the transform is the identity

The state which is controllable and observable is $x_1$

The state which is controllable and not observable is $x_2$

The state which is not controllable and observable is $x_3$

\end{enumerate}

\noindent MATLAB Result

\begin{figure}[h]
    \centering
    \includegraphics[width=0.4\textwidth]{matlab_output.png}
    \caption{MATLAB Output}
    \label{fig:current_trans}
\end{figure}

\newpage

\noindent MATLAB Code

\noindent \texttt{a2.m}

\begin{minted}{matlab}
clc; clear;

%%% Problem 3 %%%%

% a)

A = [2,-1;1,0];
B = [1;0];
C = [1,-1];

kd_a = kalman(A,B,C);
disp("3 a) T inverse")
disp(kd_a.T_inv)
disp("transformed state vector entries")
disp("    zco   zc_o  z_co  z_c_o")
disp(kd_a.z_dim)

% b)

A = [1,0,0;0,1,1;1,0,-1];
B = [0;0;1];
C = [0,0,1];

kd_b = kalman(A,B,C);
disp("3 b) T inverse")
disp(kd_b.T_inv)
disp("transformed state vector entries")
disp("    zco   zc_o  z_co  z_c_o")
disp(kd_b.z_dim)
\end{minted}

\noindent \texttt{kalman.m}

\begin{minted}{matlab}
% <strong>kalman</strong> - Kalman Decomposition
%   Author: James Marx
%   Edited: 2025.02.20
%
%   This function returns the Kalman Decomposition of a state space model.
%
%   Additional computations are made in an attempt to closer match an
%   answer which would be computed by hand - such as keeping basis vectors
%   positive and not defaulting to normalized vectors
%
%   Syntax
%   kd = <strong>kalman</strong>(A,B,C)
%
%   Input Arguments
%   <strong>A</strong> - State matrix
%   <strong>B</strong> - Input matrix
%   <strong>C</strong> - Output matrix
%
%   Output Arguments
%   <strong>kd.A</strong> - Coordinate transformed state matrix
%   <strong>kd.B</strong> - Coordinate transformed input matrix
%   <strong>kd.C</strong> - Coordinate transformed output matrix
%   <strong>kd.T_inv</strong> - Coordinate transform matrix
%   <strong>kd.z_dim</strong> - Array for dimension of transformed state variables
%   [zco    zc_o    z_co    z_c_o]
%   zco   - controllable and observable
%   zc_o  - controllable and not observable
%   z_co  - not controllable and observable
%   z_c_o - not controllable and not observable
function KD = kalman(A,B,C)
    Cb = ctrb(A,B);
    Ob = obsv(A,C);

    % find basis for Image Space of Cb
    [R, pivot_cols] = rref(Cb);
    basis_Im = R(:,pivot_cols);
    basis_Im_comp = null(basis_Im');

    % find basis for Kernel Space of Ob 
    basis_Ker = null(Ob);
    basis_Ker = normalizeOne(basis_Ker);
    basis_Ker_comp = normalizeOne(null(basis_Ker'));

    % find bases for coordinate transform matrix subspaces
    Tc_o = intersection_basis(basis_Im, basis_Ker);
    T_co = intersection_basis(basis_Im_comp, basis_Ker_comp);
    T_c_o = intersection_basis(basis_Im_comp, basis_Ker);
    Tco = intersection_basis(basis_Im, basis_Ker_comp);
    
    T_inv = [Tco, Tc_o, T_co, T_c_o];
    
    KD.A = T_inv\A*T_inv;
    KD.B = T_inv\B;
    KD.C = C*T_inv;
    KD.T_inv = T_inv;
    KD.z_dim = [size(Tco,2), size(Tc_o,2), size(T_co,2), size(T_c_o,2)];
end

% Computes a basis for the intersection of two subspaces
function B = intersection_basis(A, B)
    N = null([A, -B]);
    if (~isempty(N))
        N = normalizeOne(N);
        B = A * N(1:size(A,2), :);
        % Ensure the basis vectors are positive (if possible)
        positiveBasis = B;
        for i = 1:size(positiveBasis, 2)
            % If the first non-zero element is negative, flip the sign of the vector
            if any(positiveBasis(:, i) < 0)
                positiveBasis(:, i) = -positiveBasis(:, i);
            end
        end
        B = positiveBasis;
    else
        B = [];
    end
end

% normalize smallest value to 1 - keeping signs
function normA = normalizeOne(A)
    nonZeroElements = A(A ~= 0); % Extract non-zero elements
    smallestNonZero = min(nonZeroElements); % Find the smallest non-zero element
    normA = A / abs(smallestNonZero);
end
\end{minted}

\end{document}